\def\version{1}
\problemname{Bricks}

Josefine is playing a tetris like game called bricks. The game takes place in a rectangular grid
with $6$ columns $\times \; 8$ rows. A \textit{brick} takes up a $1 \times 1$ slot in the grid. Initially the grid is empty.
A \textit{brick} formation is a rectangle where some parts are filled with bricks and the rest is air.
The following is an example of a $4 \times 3$ brick formation where $\#$ represents bricks and $\_$
represents air:

\noindent
\begin{lstlisting}[basicstyle=\ttfamily]
  #_##
  ##__
  #__#
\end{lstlisting}

The game takes place in $N$ rounds. In each round, the player is shown a brick formation that
she must decide where (horisontally) to drop from the top of the grid. When dropping a brick
formation, each brick will indepedently fall down in a vertical line, and land either on the
bottom of the grid or directly on top of another brick (from the same formation or from
earlier rounds). Since the bricks fall indepedently, there will be no air holes between bricks in
a column afterwards (this is unlike tetris). Before dropping the brick formation, the player
may rotate it $0$, $90$, $180$, or $270$ degrees. The brick formation must be dropped such that all
bricks land within the grid

In the end of each round, all columns in the grid with at least $3$ bricks will collapse and the
bricks are thereby removed from the grid. A round $i$ has an associated round score $s_i$. Let $b_i$
be the number of collapsed bricks in a round $i$, the player then gets $b_i \cdot s_i$ points in that
round.

The goal of the game is to maximize the score over all rounds (ie. maximize $\Sigma_{i=1}^{N} b_i s_i$). Help
Josefine by writting a program that given the $N$ brick formations and round scores
computes the maximum possible score one can get.

\section*{Input}
\noindent
The first row of input contains the integer, $N$ ($1 \leq N \leq 300$), the number rounds.

Afterwards follow the information for each of the $N$ rounds. The first line of each round contains 
the integeres $w_i, h_i, s_i$ ($1 \leq w_i, h_i \leq 6$, $0 \leq s_i \leq 10000$), the width and height
of the brick formation of round $i$, and the round score for round $i$.
The following $h_i$ lines each contain a string of length $w_i$, consisting of $\#$ (bricks) or $\_$ (air), 
describing the brick formation for round $i$. The rectangle will always be the
smallest possible rectangle that covers all bricks in the formation

\section*{Output}
\noindent
Output an integer, the maximum possible score.

\section*{Points}
\noindent
Your solution will be tested on a set of test groups, each worth a number of points.
To get the points for a test group you need to solve all test cases in the test group.

\noindent
\begin{tabular}{| l | l | p{12cm} |}
  \hline
  Group & Points & Limits \\ \hline
  $1$    & $30$       &  $N \leq 5$  \\ \hline 
  $2$    & $70$       &  No additional constraints \\ \hline
\end{tabular}

\section*{Explanation of sample 1}
\noindent
If we simply drop the first brick formation as long to the left as possible without rotating it we
get:

\noindent
\begin{lstlisting}[basicstyle=\ttfamily]
  ______
  ______
  ______
  ______
  ______
  ______
  #_____
  ##____
\end{lstlisting}

If we then rotate the second brick formation $90$ degrees counter clockwise and drop it as
long to the left as possible we get: (Xs mark collapsed bricks - they will be gone when the
next round starts).

\noindent
\begin{lstlisting}[basicstyle=\ttfamily]
  ______
  ______
  ______
  ______
  X_____
  X_____
  X#____
  X#____
\end{lstlisting}

Since the round score in round $2$ is $4$, we obtain $4 \cdot 4 = 16$ points from this. Finally, we rotate
the last brick formation $180$ degrees, and drop it second most to the left, and get:

\noindent
\begin{lstlisting}[basicstyle=\ttfamily]
  ______
  ______
  ______
  ______
  _X____
  _X_X__
  _X_X__
  _X#X__
\end{lstlisting}

The last round score is $2$ and thus we obtain $2 \cdot 7 = 14$ points in this round. In total we got
$0 + 16 + 14 = 30$ points. This is optimal.