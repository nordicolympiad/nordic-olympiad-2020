\def\version{1}
\problemname{Christmas Gifts}

At the Nordic Olympic Institute (NOI) the $S$ students have a tradition of exchanging gifts with
their friends at christmas. More precisely, if $A$ and $B$ are friends either $A$ gives $B$ a gift or
$B$ gives $A$ a gift.

Last christmas there was a big scandal at NOI because some of the students received a lot
of gifts without giving any and some students gave a lot of gifts without receiving any. NOI
needs your help to make this christmas gift exchange more fair. You need to decide who
should give gifts to whom, ie. for each friendship between $A$ and $B$, you have to decide if $A$
should give a gift to $B$ or $B$ should give a gift to $A$.

Let $g_i$ be the number of gifts student $i$ gives, and $r_i$ the number of gifts student $i$ receives.
The NOI administration has decided a fair gift exchange is one that minimizes the \textit{unfairness}
score $\Sigma_{i=1}^{S} |g_i-r_it|$.

Given the list of $F$ friendships, compute the minimum possible unfairness score and for each
friendship write out who should give a gift to whom. Due to GDPR concerns, all students
have been numbered from $1,\ldots,S$ instead of using their names.

\section*{Input}
\noindent
The first row of input contains the integers $S, F$ ($2 \leq S \leq 10^5$, $1 \leq F \leq 2 \cdot 10^5$), the number
of students and the number of friendships.

The following $F$ lines each contain two integers $A, B$ ($1 \leq A, B \leq S$), meaning that $A$ and $B$ are friends.
All friendships are mutual and will appear once in the input.

\section*{Output}
\noindent
On the first row, output the minimum possible unfairness score.
On each of the following $F$ lines, output integers $A$ and $B$, meaning that $A$ gives a gift to $B$.
You can output these in any order, but each friendship must be output exactly once.

\section*{Points}
\noindent
Your solution will be tested on a set of test groups, each worth a number of points.
To get the points for a test group you need to solve all test cases in the test group.

\noindent
\begin{tabular}{| l | l | p{12cm} |}
  \hline
  Group & Points & Limits \\ \hline
  $1$    & $15$       &  $S \leq 10$, $F \leq 20$  \\ \hline 
  $2$    & $15$       &  $S \leq 500$, $F \leq 10000$ \\ \hline
  $3$    & $20$       &  All students have an even number of friends \\ \hline
  $4$    & $50$       &  No additional constraints \\ \hline
\end{tabular}

\section*{Explanation of sample 1}
\noindent
In this example we have $4$ students and $5$ friendships. The shown solution is not unique - any
correct solution will be accepted.
